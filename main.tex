\documentclass[12pt]{article}

% Packages
\usepackage{amsmath, amssymb}  % For mathematical symbols
\usepackage{amsthm}            % For theorem environments
\usepackage{geometry}          % For page layout
\usepackage{fancyhdr}          % For header and footer
\usepackage{xcolor}            % For colored text
\usepackage{enumitem}          % For custom lists
\usepackage[colorlinks=true, linkcolor=black, citecolor=blue, urlcolor=blue]{hyperref}

\usepackage{tikz}        % For creating the custom shapes and drawing the box
\usepackage{graphicx}    % Recommended if you’re working with images and figures


% Page layout
\geometry{letterpaper, margin=1in}
\setlength{\parindent}{0pt}
\setlength{\parskip}{8pt}

% Header and Footer
\pagestyle{fancy}
\fancyhf{}
\fancyhead[L]{\textbf{Lecture Notes}}
\fancyhead[R]{\thepage}
\fancyfoot[L]{}
\fancyfoot[C]{}
\fancyfoot[R]{}

% Title and author
\title{\textbf{Mathematics for Computer Science}}

\author{MIT - 6.042J}
\date{\today}

% Watermark settings for name
\usepackage{enumitem}          % For custom lists
\usepackage{background}        % For watermark
\backgroundsetup{
  position=current page.south east,
  angle=0,
  nodeanchor=south east,
  color=black!30,
  scale=0.8,
  vshift=10pt,
  hshift=-15pt,
  contents={\small \href{https://www.linkedin.com/in/bilalnaseem96/}{Bilal Naseem}} % Replace with your name
}



% Custom environments
\newtheorem{theorem}{Theorem}[section]
\newtheorem{definition}{Definition}[section]
\newtheorem{example}{Example}[section]
\newenvironment{important}{\color{red}\textbf{Important:} }{}

% Custom commands
\newcommand{\note}[1]{\textcolor{blue}{\textbf{#1}}} % Command for notes
\newcommand{\exampletitle}[1]{\textbf{#1}}
\newcommand{\mybox}[3]{
    \begin{figure}[h]
        \centering
        \begin{tikzpicture}
            \node[anchor=text,text width=\columnwidth-1.2cm, draw, rounded corners, line width=1pt, fill=#2!5, inner sep=5mm] (big) {\\#3};
            \node[draw, rounded corners, line width=.5pt, fill=#2!30, anchor=west, xshift=5mm] (small) at (big.north west) {#1};
        \end{tikzpicture}
    \end{figure}
}
\newcommand{\insertimage}[2]{
    \begin{figure}[h]
        \centering
        \includegraphics[width=#1\textwidth]{#2}  % Width can be specified
        % No caption or label included
    \end{figure}
}


\begin{document}

% Title Page
\maketitle

% Notations Page


\tableofcontents
\newpage

\section*{Notations}
\begin{itemize}
    \item $\forall$ : For all, for every possible choice of n
    \item $\mathbb{N}$ : Set of Natural Numbers i.e. $\{0, 1, 2, 3, 4 \ldots\}$
    \item $\mathbb{R}$ : Set of all real numbers
    \item $\mathbb{Z}$ : Set of all integers
    \item $\mathbb{P}$: Prime number is a number that is only divisible by 1 and itself.
    \item $\exists$ : There exists, there is at least one possible choice of n.
    \item $\nexists$ : There does not exist, there is no possible choice of n.
    \item $\sin(\theta)$ : Sine function
    \item $\cos(\theta)$ : Cosine function
    \item $\nabla f$ : Gradient of function $f$
    \item $\int_a^b f(x) \, dx$ : Definite integral of $f$ from $a$ to $b$
          % Add any additional notation you want to include here
\end{itemize}



\newpage

% Main Content
\section{Introduction to Proofs}
A \textit{Proof} is a method for ascertaining (find (something) out for certain; make sure of) the \textbf{truth}. Examples of ways in which we ascertain proof:
\begin{enumerate}
    \item Experimentation and observation (bedrock of physics)
    \item Counterexamples
    \item Sampling (repeating a process a large number of times)
    \item Judge/Jury system
    \item Religion
    \item Word of boss (authority)
    \item Inner conviction
\end{enumerate}

\subsection{Mathematical Proof}
A \textit{Mathematical Proof} is a verification of a \textbf{Proposition} by a chain of \textbf{logical deductions} from a set of \textbf{axioms}.

\subsection{Proposition}
A Proposition is a statement that is either true or false. Examples:
\begin{itemize}
    \item $2+3=5$
    \item $\forall n\in  \mathbb{N}, n^2+n+41$ is a prime number.
\end{itemize}

\subsubsection{Predicate}
$$
    \forall n \in \mathbb{N}, \underbrace{n^2 + n + 41}_{\text{Predicate}}
$$
A Predicate is a Proposition whose truth is determined by the values of its variables.

In order to determine if this Proposition is true, we need to make sure that this Predicate is true for every Natural Number $\mathbb{N}$. Lets verify the Predicate by plugging in some values of Natural Numbers. The predicate if false, even though it worked for the first 40 numbers.

\[
    \begin{array}{|c|c|c|}
        \hline
        n      & n^2 + n + 41 & \text{Prime} \\
        \hline
        0      & 41           & \checkmark   \\
        1      & 43           & \checkmark   \\
        2      & 47           & \checkmark   \\
        3      & 53           & \checkmark   \\
        \vdots & \vdots       & \vdots       \\
        20     & 461          & \checkmark   \\
        \vdots & \vdots       & \vdots       \\
        39     & 1601         & \checkmark   \\
        40     & 1681=41^2    & \times       \\
        41     & .            & \times       \\
        \hline
    \end{array}
\]

\subsection{Euler's Theorem}

\begin{example}
    $a^4 + b^4 + c^4 = d^4$ has no positive integer solutions.
\end{example}

This proposition was unsolved for centuries. It was conjectured to be true by Euler in 1769. It was disproved recently using very big numbers. Hence the correct Proposition would be :

$$
    \exists \quad a, b, c, d \in \mathbb{N}^+, \quad \underbrace{a^4 + b^4 + c^4 = d^4}_{\text{Predicate}}
$$
We care about such these proof problems because they involve \textit{Factoring large integers}. It is a way to break crypto systems like RSA which are used for everything we do electronically today.

\begin{example}
    The regions in any map can be coloured in 4 colours such that adjacent regions have different colours.
\end{example}

\section{Induction}
% \begin{important}
%     This section is for important notes or remarks to highlight key points.
% \end{important}

% \note{ljlkj}

\end{document}
